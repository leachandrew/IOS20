\documentclass{article}
\usepackage{booktabs, graphicx, hyperref, fontspec}
\usepackage{sectsty}
\allsectionsfont{\sffamily}
\usepackage[margin=1in]{geometry}
\hypersetup{
  colorlinks = true,
  urlcolor = cyan,
 }
 \providecommand{\tightlist}{%
  \setlength{\itemsep}{0pt}\setlength{\parskip}{0pt}}
\newcommand*{\authorfont}{\fontfamily{phv}\selectfont}
\usepackage[]{Fira Sans}

\begin{document}

\sffamily

\centerline{\Huge Economics of Development}

\vspace{3 mm}

\centerline{\large Dr.~Ryan Safner}
\vspace{2 mm}
\centerline{\large \href{http://devf19.classes.ryansafner.com}{devF19.classes.ryansafner.com}}

\vspace{5 mm}

\begin{tabular}{@{}p{3.5in}p{3.5in}}           
\textbf{Course}: ECON 317 - Fall 2019  & \textbf{Email:}  \href{mailto:safner@hood.edu}{\nolinkurl{safner@hood.edu}}\\
\textbf{Room}: Rosenstock Hall 215 & \textbf{Office:}  Rosenstock Hall 118\\
\textbf{Meets}: MW 11:30 A.M--12:45 P.M. & \textbf{Hours:} TuTh 10:30--11:30 A.M.\\ 
\end{tabular}

\vspace{5 mm}

\hrule


\begin{quote}
``The consequences for human welfare involved in questions like these
are simply staggering: Once one starts to think about them, it is hard
to think about anything else. -- Robert Lucas, Nobel Laureate in
Economics 1995''
\end{quote}

\textbf{Economic Development} asks and attempts to answer the
fundamental question posed by Adam Smith in 1776: \emph{why are some
countries wealthy, and others poor?}. Before ``economics'' emerged as
its own discipline in the 20\textsuperscript{th} century, and adopted
more rigorous mathematical trappings, the exploration of \emph{political
economy} attempted to answer this question: political, social, and
cultural institutions and histories had as much to offer as the division
of labor and market exchanges in answering the challenge. In the
aftermath of WWII, economists in ``the First World'' began more
consciously studying how to promote development in ``the Third World'',
while policymakers built international institutions (the IMF, the World
Bank, the Bretton-Woods financial systems) aimed at securing peace and
outwardly developing what they say as the ``Third World.'' In order to
grapple with these key questions, we will examine a mixture of political
economy and economic history to understand the role of political,
cultural, and social institutions in directing economic activity towards
prosperity or towards ruin. This course, like the professionals dealing
with the big questions, will offer many suggestions but fewer
``correct'' or concrete answers than you may be used to. You should come
to this course as a willing participant in the ongoing conversation.

The economics of development combines core themes and models of
macroeconomics (growth theory, macroeconomic stability and policy) with
core principles of microeconomics (price theory), and as such, the
\textbf{prerequisites} for this course are \textbf{either ECON 205 or
ECON 206}.

This course will be a hybrid of formal lecture, hands-on activities, and
student presentations of scholarly articles, with exams and a writing
writing assignment serving as the prime means of evaluation.

\hypertarget{course-objectives}{%
\section{Course objectives}\label{course-objectives}}

\textbf{By the end of this course,} you will:

\begin{itemize}
\tightlist
\item
  Explain how the development community measures economic development
\item
  Interpret regression tables in the empirical literature in development
\item
  Demonstrate different theories of economic development
\item
  Explain why various policies aimed at promoting development have
  failed
\item
  Describe essential conditions for successful development
\item
  Discuss the broad economic history of ``the West'', several key
  ``Emerging Markets'' (such as Russia, China, Korea, etc.), and several
  other case studies of developing countries
\end{itemize}

Given these objectives, this course fulfills two of the learning
outcomes for
\href{https://www.hood.edu/academics/departments/george-b-delaplaine-jr-school-business/student-learning-outcomes}{the
George B. Delaplaine, Jr.~School of Business} Economics B.A. program:

\begin{itemize}
\tightlist
\item
  Apply economic reasoning and models to understand and analyze problems
  of public policy {[}\ldots{]}
\item
  Demonstrate effective oral and written communications skills for
  personal and professional success{[}\ldots{]}
\end{itemize}

\textbf{My standard disclaimer:} This class may challenge many of your
existing beliefs and conceptions about how the world works, and how it
should work. This is the most important and exciting part of a liberal
arts education. This does \emph{not} mean that I want to make you to see
everything ``my way.'' In fact, if you come out of this class thinking
exactly like me, then I have probably failed you as a teacher. To the
best of my ability, I keep my opinions to myself unless relevant for
purposes of discussion, and I respect and invite you to reach your own
conclusions on all matters.

\textbf{Content warning:} this class will cover sensitive political and
cultural topics and compel you to grapple with countries, cultures, and
viewpoints very different from your own. To put it mildly, these topics
may include themes of violence, slavery, imperialism, and different
ideologies inherently wrapped up in the tragic history of both the
developed and developing world.

If at any point you find yourself struggling in this course for any
reason, please come see me. Do not suffer in silence! Coming to see me
for help does not diminish my view of you, in fact I will hold you in
\emph{higher} regard for understanding your own needs and taking charge
of your own learning. There are also a some fantastic resources on
campus, such as the
\href{http://www.hood.edu/campus-services/academic-services/index.html}{Center
for Academic Achievement and Retention (CAAR)} and the
\href{http://www.hood.edu/library/}{Beneficial-Hodson Library}.

See my \href{http://devf19.classes.ryansafner.com/reference\#tips}{tips
for success in this course}.

\hypertarget{course-materials}{%
\section{Course materials}\label{course-materials}}

You can find all course materials at my academic website dedicated for
(all sections of) this course
\href{http://devf19.classes.ryansafner.com}{devF19.classes.ryansafner.com}.

My lecture slides will be shared with you, and serve as your primary
resource, but our main ``textbook'' below is \textbf{recommended} as the
next best resource and will be available from the campus bookstore.

\hypertarget{books}{%
\subsection{Books}\label{books}}

My lecture slides (made available to you) are the primary resource for
the material and the best guide to prepare for assignments. There are
two books that we will roughly be following in parallel, both available
at the bookstore (or you can find them on Amazon, ebay, etc). You may
choose to purchase used or old versions, but be aware that content and
ordering may slightly vary across versions.

\begin{enumerate}
\def\labelenumi{\arabic{enumi}.}
\tightlist
\item
  Acemoglu, Daron and James A. Robinson, 2008, \emph{Why Nations Fail:
  The Origins of Power, Prosperity, and Poverty,} New York: Crown
  Business
\item
  Easterly, William, 2000, \emph{The Elusive Quest for Growth:
  Economists' Adventures and Misadventures in the Tropics,} Cambridge,
  Mass: MIT Press
\end{enumerate}

Both books are landmarks in the study of economic development by
renowned development economists and are written for a popular audience.
These books should be easily readable and affordable -- you could buy
and read them at the airport or the beach (should you be nerdy enough
like me). Both are listed as \textbf{required} in the bookstore, but
feel free to get them elsewhere.

The first book is something like ``our textbook'' for the course, as it
outlines many of the key topics that we cover this semester. We will
have frequent readings from it, but my coverage of topics and sequencing
will be different from the book. It is one of my favorite books due to
the central role that different institutions play in determining the
variation among countries today.

The second book is older, but aptly describes the history of development
economics as a field, and is a brilliant and relentless narrative of all
of the policies, fads, and politics of the development community and how
many of them went horribly wrong.

\hypertarget{articles}{%
\subsection{Articles}\label{articles}}

Throughout the course, I will post both required and supplemental
(non-required) readings that enrich your understanding for each topic.
Check \emph{frequently} for announcements and updates to assignments,
readings, and grades.

\hypertarget{assignments-and-grades}{%
\section{Assignments and grades}\label{assignments-and-grades}}

\textbf{You can find descriptions} for all the assignments on the
\href{http://devf19.classes.ryansafner.com/assignments/}{assignments
page}.

\begin{center}

\begin{tabular}{lll}
\toprule
 & Assignment & Percent\\
\midrule
n & Participation (Average) & 25\%\\
1 & Country Profile & 5\%\\
2 & Short Paper & 20\% each\\
1 & Final & 30\%\\
\bottomrule
\end{tabular}
\end{center}

All grades are based on the following traditional scale:

\begin{center}

\begin{tabular}{llll}
\toprule
Grade & Range & Grade1 & Range1\\
\midrule
A & 93–100\% & C & 73–76\%\\
A− & 90–92\% & C− & 70–72\%\\
B+ & 87–89\% & D+ & 67–69\%\\
B & 83–86\% & D & 63–66\%\\
B− & 80–82\% & D− & 60–62\%\\
\addlinespace
C+ & 77–79\% & F & < 60\%\\
\bottomrule
\end{tabular}
\end{center}

These grades are firm cutoffs, but I do of course round upwards
(\(\geq 0.5\)) for final grades. A necessary reminder, as an academic, I
am not in the business of \emph{giving} out grades, I merely report the
grade that you \emph{earn}. I will not alter your grade unless you
provide a reasonable argument that I am in error (which does happen from
time to time).

\hypertarget{policies-and-expectations}{%
\section{Policies and Expectations}\label{policies-and-expectations}}

This syllabus is a contract between you, the student, and me, your
instructor. It has been carefully and deliberately thought out\footnote{A
  syllabus can and will be used as a legal document for disputes tried
  at a court of law. Ask me how I know.}, and I will uphold my end of
the agreement and expect you to uphold yours.

In the language of game theory, this syllabus is my commitment device. I
am a very understanding person, and I know that exceptions to rules
often need to be made for students. However, to be \emph{fair} to
\emph{all} students the syllabus artificially constrains my ability to
make exceptions at a whim for anyone. This prevents clever students from
exploiting my congenial personality at everyone else's expense. Please
read and familiarize yourself with the course policies and expectations
of you. Chances are, if you have a question, it is answered herein.

\hypertarget{attendance-and-participation}{%
\subsubsection{Attendance and
Participation}\label{attendance-and-participation}}

I expect you to attend class and to come having already done the reading
assigned for that day. I will remind you in class and possibly through
Blackboard or email which readings I want you to read for the next
class. You are all adults and I will treat you as such. I do not take
attendance, nor do I grade formally for participation but I strongly
recommended you attend class and participate for your sake and the sake
of your classmates. If you are too distracted or are not prepared to
learn, I suggest you stay home, where you can check Facebook more
efficiently. I reserve the right to boost the final grades of students
that I believe have made consistent, quality contributions above and
beyond their peers in class conversations by up to 2.5 points.

\hypertarget{devices}{%
\subsubsection{Devices}\label{devices}}

You are allowed to have and use laptops and tablets in the classroom. I
will not stop you, but I strongly discourage you from using these to
take notes (see Tips for Success). As a courtesy to myself and to
others, do not use your phone in class. I reserve the right to embarrass
you in front of everyone if you do so.

\hypertarget{absences-and-make-ups}{%
\subsubsection{Absences and Make-Ups}\label{absences-and-make-ups}}

You generally do \emph{not} need to let me know if you are unable to
make class, \emph{unless} it is on the day of an exam. It will however,
be your responsibility to acquire the notes from a classmate for any
missed classes. If you are unable to attend an exam for a legitimate
reason (e.g.~sports/club events, traveling, illness, family issues),
please notify me at least \emph{one week} in advance, and we will
schedule a make-up exam date. If you are ill or otherwise unable to
attend on the day of the exam, contact me ASAP to make arrangements.
Failure to do so, including desperate attempts to make arrangements only
\emph{after} the absence will result in a grade of 0 and little
sympathy. I reserve the right to re-weight other assignments for
students who I believe are legitimately unable to complete a particular
assignment.

\hypertarget{late-assignments}{%
\subsubsection{Late Assignments}\label{late-assignments}}

I will accept late assignments, but will subtract a specified amount of
points as a penalty. See individual assignment descriptions for the
amount of points taken off (as it varies by assignment). If an answer
key is posted before you turn in your assignment, the maximum grade you
can earn is an 80. Even if it is the last week of the semester, I
encourage you to turn in late work: some points are better than no
points!

\hypertarget{grading}{%
\subsubsection{Grading}\label{grading}}

I will try my best to post grades on Blackboard's Grading Center and
return graded assignments to you within about one week of you turning
them in. There will be exceptions. Where applicable, I will post answer
keys once I know most homeworks are turned in (see Late Assignments
above for penalties). Blackboard's Grading Center is the place to look
for your most up-to-date grades. You will also be given an Excel
spreadsheet template where you can calculate your overall grade and
forecast ``what if'' scenarios.

\hypertarget{email-accounts}{%
\subsubsection{Email Accounts}\label{email-accounts}}

Students must regularly monitor their Hood email accounts to receive
important college information, including messages related to this class.
Email through the Blackboard system is my main method of communicating
announcements and deadlines regarding your assignments. \textbf{Please
do not reply to any automated Blackboard emails - I may not recieve
it!}. My Hood email
(\href{mailto:safner@hood.edu}{\nolinkurl{safner@hood.edu}}) is the best
means of contacting me. I will do my best to respond within 24 hours. If
I do not reply within 48 hours, do not take it personally, and
\emph{feel free to send a follow up email} in the very likely event that
I genuinely did not see your original message.

\hypertarget{office-hours}{%
\subsubsection{Office Hours}\label{office-hours}}

I am generally in my office Monday-Thursday during ``normal business
hours.'' You are always welcome to walk-in and chat about class,
college, careers, or anything at all. Please do try to use the official
office hours stated at the head of the syllabus if possible. If you need
to meet at a different time, I request that you send me an email or let
me know after class so I know when to expect you. If you want to go over
material from class, please have \emph{specific} questions you want help
with. I am not in the business of giving private lectures (particularly
if you missed class without a valid excuse).

Watch the excellent and accurate video explaining office hours (on
website syllabus page):

\hypertarget{enrollment}{%
\subsubsection{Enrollment}\label{enrollment}}

Students are responsible for verifying their enrollment in this class.
The last day to add or drop this class with no penalty is
\textbf{Tuesday, September 4}. Be aware of
\href{https://www.hood.edu/offices-services/registrars-office/academic-calendar}{important
dates}.

\hypertarget{honor-code}{%
\subsubsection{Honor Code}\label{honor-code}}

Hood College has an Academic Honor Code which requires all members of
this community to maintain the highest standards of academic honesty and
integrity. Cheating, plagiarism, lying, and stealing are all prohibited.
All violations of the Honor Code are taken seriously, will be reported
to appropriate authority, and may result in severe penalties, including
expulsion from the college. See
\href{http://hood.smartcatalogiq.com/en/2016-2017/Catalog/The-Spirit-of-Hood/The-Academic-Honor-Code-and-Code-of-Conduct}{here}
for more detailed information.

\hypertarget{van-halen-and-mms}{%
\subsubsection{Van Halen and M\&Ms}\label{van-halen-and-mms}}

When you have completed reading the syllabus, email me a picture of the
band Van Halen and a picture of a bowl of M\&Ms.~If you do this
\emph{before} the date of the first exam, you will get 3 bonus points on
the exam. Yes, this is real.

\hypertarget{accessibility-equity-and-accommodations}{%
\subsubsection{Accessibility, Equity, and
Accommodations}\label{accessibility-equity-and-accommodations}}

College courses can, and should, be challenging and bring you out of
your comfort zone in a safe and equitable environment. If, however, you
feel at any point in the semester that certain assignments or aspects of
the course will be disproportionately uncomfortable or burdensome for
you due to any factor beyond your control, please come see me or email
me. I am a very understanding person and am happy to work out a solution
together. I reserve the right to modify and reweight assignments at my
sole discretion for students that I belive would legitimately be at a
disadvantage, through no fault of their own, to complete them as
described.

If you are unable to afford required textbooks or other resources for
any reason, come see me and we can find a solution that works for you.

This course is intended to be accessible for all students, including
those with mental, physical, or cognitive disabilities, illness,
injuries, impairments, or any other condition that tends to negatively
affect one's equal access to education. If at any point in the term, you
find yourself not able to fully access the space, content, and
experience of this course, you are welcome to contact me to discuss your
specific needs. I also encourage you to contact the
\href{https://www.hood.edu/academics/josephine-steiner-center-academic-achievement-retention/accessibility-services}{Office
of Accessibility Services} (301-696-3421). If you have a diagnosis or
history of accommodations in high school or previous postsecondary
institutions, Accessibility Services can help you document your needs
and create an accommodation plan. By making a plan through Accessibility
Services, you can ensure appropriate accommodations without disclosing
your condition or diagnosis to course instructors.

\hypertarget{schedule}{%
\section{Schedule}\label{schedule}}

\textbf{You can find a full schedule} with resources for each class
meeting on the
\href{http://devF19.classes.ryansafner.com/schedule/}{course website
schedule page}.

\end{document}